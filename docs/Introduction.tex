\chapter{はじめに}
\section{背景}
近年,時系列データと呼ばれるデータへの解析が盛んになっている.
時系列データとは,時間的な順序を伴いながら観測されるデータのことである.
分かりやすい例だと気温や株価などのデータであるが,近年では自身の体重や1日で歩いた距離などの身近な記録さえも時系列データとして収集できる時代になり,時系列データの解析に注目が集まっている.
これら時系列データに対して解析をする上で基本的な技術の一つとして厳密文字列照合がある.

厳密文字列照合~\cite{exact}とは,アルファベット上のテキスト$T$と呼ばれる文字列の中からパターン$P$と呼ばれる探したい文字列と完全に一致している文字列の位置を全て探す照合である.
例えば,時系列データでの厳密文字列照合では,$T=(20,10,50,40,60)$,$P=(20,10,50)$とすると,テキストの先頭にパターンと一致する文字列を見つけることができる.
しかし,時系列データ分析では,パターン$P$と完全に一致している部分文字列$T[1:3]$を探すだけではなく,$P$と``似ている''部分文字列$T[3:5]=(50,40,60)$を探すことも意味がある.
そのため,パターンと``似ている''文字列をテキストから探す照合が様々あり,それらの一つとして順序保存照合やデカルト木照合がある.

順序保存照合~\cite{op}とは,全順序アルファベット上のテキスト文字列$T$の中からパターン文字列$P$の要素の順位と一致する部分文字列の位置を全て探す照合で,例えば$T=(33,25,36,18,45,30,49,26),P=(17,10,19,6,24,15,27)$が与えられた時,$P$の各要素の順位の列は$(4,2,5,1,6,3,7)$であり,これと同じ順位の列である$T[1:7]=(33,25,36,18,45,30,49)$を見つける.
一方,デカルト木照合~\cite{cartesian-tree}とは,テキスト文字列$T$の中からパターン文字列$P$のデカルト木と同じデカルト木となる文字列を全て探す照合である.
デカルト木とは,文字列を表す木構造であり,文字列$S_1=(17,10,19,6,24,15,27)$のデカルト木は以下の図\ref{fig:introデカルト木}のようになる.
例えば,文字列$S_1=(17,10,19,6,24,15,27)$と文字列$S_2=(29,9,21,4,23,14,27)$がデカルト木照合でマッチする文字列同士か判断する場合,以下の図\ref{fig:introデカルト木}より文字列$S_1$と$S_2$はともに同じデカルト木となる文字列なので,デカルト木照合でマッチする.
しかし,順序保存照合の場合,文字列$S_1$と$S_2$の各要素の順位の列は一致しないので,順序保存照合ではマッチしない.
このように,デカルト木照合では,順位の列が完全に一致していない文字列でも,文字列のデカルト木の形が同じでさえあれば検出できる照合なので,順序保存照合よりも寛容な照合であり,より直感的に形が似ている文字列を探すことができる照合である.

本論文で紹介する,$k$ミスマッチデカルト木問題

% 本論文で紹介する,1ミスマッチデカルト木照合とは,テキスト文字列$T$とパターン文字列$P$が与えられた時,$P$と1ミスマッチデカルト木同型となるような部分文字列を$T$から探す照合である.
% 1ミスマッチデカルト木同型とは,同じ長さの2つの文字列において,互いに同じ位置の要素を取り除いた部分列のデカルト木が一致していることを表している.
% このようなミスマッチを許容する照合は他の照合でも考えられており,テキスト長を$n$,パターン長を$m$,取り除く要素数を$k$とすると,$k$ミスマッチ厳密文字列照合はLandau~\cite{kmiss-exact}によって$O(k(m \log m + n))$時間のアルゴリズムが提案され,$k$ミスマッチ順序保存照合はGawrychowski~\cite{kmiss-op}によって$O(n(\log \log m + k \log \log k))$時間のアルゴリズムが提案されている.

\indent 本論文では,$k$ミスマッチデカルト木問題を定義し,$O(m^2k^2)$時間で解くアルゴリズムを提案した.
% \begin{figure*}[htbp]
%   \begin{minipage}[b]{0.5\linewidth}
%     \centering
%     \includegraphics[width=\linewidth]{fig/intro-s1-crop.pdf}
%     % \caption{文字列$S_1=(17,10,19,6,24,15,27)$のグラフとデカルト木}
%   \end{minipage}
%   \begin{minipage}[b]{0.5\linewidth}
%     \centering
%     \includegraphics[width=\linewidth]{fig/intro-s2-crop.pdf}
%   \end{minipage}
%   \caption{文字列$S_1=(17,10,19,6,24,15,27)$と文字列$S_2=(29,9,22,4,23,14,27)$のグラフとデカルト木}
%   \label{fig:introデカルト木}
% \end{figure*}

\section{構成}
本論文の構成は次のとおりである.
まず,第2章において表記法,$k$ミスマッチデカルト木照合の定義を述べる.
第3章において提案手法を述べる.
第4章においてまとめと今後の課題を述べる.


