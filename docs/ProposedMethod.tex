\chapter{提案手法}
$k$ミスマッチデカルト木決定問題を動的計画法を用いて,$O(m^2k^2)$時間で解けることを示す.
\section{素朴なアルゴリズム}
あとで書く
\section{動的計画法}
$DP(i,j,v)$を以下のように定義する.整数$i,j\ (1\leq i \leq j \leq m)$,整数$v\ (i\leq v \leq j)$に対して,
\begin{displaymath}
  \begin{aligned}
    DP(i,j,v) = \min(\{ & j-i+1-|I| \mid I\in L_{[1:m]}\ \land\  CT(S_{I})=CT(P_{I})\ \land \nonumber \\
                       & v\in I\ \land\ CTR(P_I)=CTR(S_I)=v \} \cup \{j-i+1\})
  \end{aligned}
\end{displaymath}
とする.\\
次の補題を用いて,$DP(i,j,v)$を計算する.
\begin{lemma}
  $DP(i,j,v)$は以下の漸化式を満たす.
  \begin{displaymath}
    \begin{aligned}
      DP(i,j,v) &= leftmiss + rightmiss\\
      leftmiss &=
        \begin{cases}
          \min\{DP(i,v-1,l) \mid l \in A\} & (\text{集合}A=\\& \{l \in \{i,\cdots, v-1\}\mid P[l]>P[v] \land S[l]>S[v]\})\\
          v-i & (\text{Aが空集合の場合})
        \end{cases}\\
      rightmiss &=
        \begin{cases}
          \min\{DP(v+1,j,r) \mid r \in B\} & (\text{集合}B=\\&\{l \in \{v+1,\cdots, j\}\mid P[l]\geq P[v] \land S[l]\geq S[v]\})\\
          j-v & (\text{Bが空集合の場合})
        \end{cases}
    \end{aligned}
  \end{displaymath}
\end{lemma}

\begin{proof}
  あとで書く
\end{proof}

以下の補題\ref{lem:kmissCTMiss}は,$k$ミスマッチデカルト木決定問題を解くためには,$DP(1,m,v)$において,全ての$v\ (1\leq v \leq m)$を調べる必要はなく,$v\in TopK(S[1:m], k+1)$だけを調べればよいことを示す.
\begin{lemma}\label{lem:kmissCTMiss}
  \begin{displaymath}
    CTMiss(S,P)\leq k \Leftrightarrow \min\{DP(1,m,v) \mid v \in TopK(S[1:m], k+1)\} \leq k
  \end{displaymath}
\end{lemma}

\begin{proof}
  あとで書く
\end{proof}

\section{提案手法の時間計算量}
補題\ref{lem:kmissCTMiss}より,集合AとBの要素$l$と$r$はそれぞれ集合$TopK(S[i:v-1], k+1)$と集合$TopK(S[v+1:j], k+1)$内の要素からなる.
任意の$i,j\ (1\leq i \leq j \leq m)$における$TopK(S[i:j],k+1)$に対しては,$O(m^2k)$時間で前処理計算する.
これにより,求める必要がある$DP(i,j,v)$は$1\leq i \leq j \leq m$,\ $v \in TopK(S[i:j], k+1)$と$l\in TopK(S[i:j], k+1)$の範囲をループすれば求めることができるので,全体の時間計算量は$O(m^2k^2)$時間となる.
提案手法の疑似コードを以下のAlgorithm\ref{alg:DP}に示す.

\begin{algorithm}
  \caption{提案手法のアルゴリズム}
  \label{alg:DP}
  \KwIn{文字列$S[1:m]$,文字列$P[1:m]$,許容ミスマッチ数$k$}
  \KwOut{$CTMiss(S,P)\leq k$かどうか True/False}
  $|S|\times |S| \times k$次元の配列を$|S|-1$で初期化\;
  $CTMiss \leftarrow |S|-1$\;
  任意の$i,j\ (1\leq i \leq j \leq m)$において,集合$TopK(S[i:j],k+1)$を前処理で計算する\;
  \For{$i \leftarrow 1$ \KwTo $m$}{
    \For{$j \leftarrow i+1$ \KwTo $m$}{
      \For{$v \in TopK(S[i:j],k+1)$}{
        $leftmiss \leftarrow v-i$\;
        \If{$v\geq 2$}{
          \For {$l \in TopK(S[i:v-1],k+1)$}{
            \If{$P[l]>P[v] \land S[l]>S[v]$}{
              $leftmiss \leftarrow \min\{DP(i,v-1,l),\ leftmiss\}$\;
            }
          }
        }
        $rightmiss \leftarrow j-v$\;
        \If{$v\leq m-1$}{
          \For {$r \in TopK(S[v+1:j],k+1)$}{
            \If{$P[r]\geq P[v] \land S[r]\geq S[v]$}{
              $rightmiss \leftarrow \min\{DP(v+1,j,r),\ rightmiss\}$\;
            }
          }
        }
        $DP(i,j,v) \leftarrow \min\{DP(i, j, v),\ leftmiss+rightmiss\}$\;
        $CTMiss \leftarrow \min\{CTMiss, DP(1,m,v)\}$\;
      }
    }
  }
  \Return{$CTMiss\leq k$}
\end{algorithm}


