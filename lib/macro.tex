%-*- coding: utf-8 -*-

% ToDo や,各自のコメント行の表示のon/offは,次の命令で切り替える.
\newif\ifdraft \drafttrue   %% 作業用
%\newif\ifdraft\draftfalse  %% 提出用

%%%%%%%%%%%%%%%%%%%% コメント用 %%%%%%%%%%%%%%%%%%

\ifdraft
% 自分の好きな色を選んで下さい
\newcommand{\shinocom}[1]{\textcolor{magenta}{[(篠原)#1]}}
\newcommand{\yoshicom}[1]{\textcolor{red}{[(吉仲)#1]}}
\newcommand{\usercom}[1]{\textcolor{blue}{[(user)#1]}}
\newcommand{\todo}[1]{{\color{red}{[ToDo: #1]}}}
\else
\newcommand{\shinocom}[1]{}
\newcommand{\yoshicom}[1]{}
\newcommand{\usercom}[1]{}
\newcommand{\todo}[1]{}
\fi

%%%%%%%%%%%%%%%%%%%% パッケージ %%%%%%%%%%%%%%%%%%

% \usepackage[dvipdfmx]{color}                   % 色の使用
\usepackage[dvipdfmx]{graphicx}                % 図表
\usepackage{amssymb}                           % 特殊文字が使用可能となる
\usepackage{amsmath}                           % 数式を書く人の定番.とりあえず入れておく的な
\usepackage{amsthm}                            % 定理環境など 
% \usepackage{comment}                           % 複数行コメントアウト
% \usepackage{dcolumn}                           % 表中で小数点揃えなどを実現する
% \usepackage[setpagesize=false]{hyperref}       % PDFファイルにハイパーリンクを埋め込む
% \usepackage{multirow}                          % 表の縦結合
% \usepackage{subfigure}                         % 図の中にサブ(a)(b)とか作るやつ
% \usepackage{colortbl}                          % 表の網掛け
% \usepackage{graphicx}                          % 文字列の縦圧縮
% \usepackage{url}                               % URLを使うとき
\usepackage{tikz}                              % 美しい図を書こう
\usepackage[ruled, linesnumbered]{algorithm2e} % 擬似コードを書きやすい
% \SetArgSty{textup}                             % 上記 algorithm2e を使うときはこの設定がおすすめ


%%%%%%%%%%%%%%%%%%%% マクロ定義 %%%%%%%%%%%%%%%%%%

%\newcommand{\qed}{\hfill\square}

\newcommand{\mcal}[1]{\mathcal{#1}}
\newcommand{\msf}[1]{\mathsf{#1}}
\newcommand{\mrm}[1]{\mathrm{#1}}
\newcommand{\mtt}[1]{\mathtt{#1}}
\newcommand{\tsf}[1]{\textsf{#1}}
\newcommand{\tsc}[1]{\textsc{#1}}
\newcommand{\ttt}[1]{\texttt{#1}}
\newcommand{\lrangle}[1]{\langle #1 \rangle}
%\newcommand{\lrangle}[1]{\left\langle #1 \right\rangle} % お好みで

%%%%%%%%%%%%%%%%%%% 定理環境定義 %%%%%%%%%%%%%%%%%%
\theoremstyle{definition}
\newtheorem{definition}{定義}
\newtheorem{theorem}{定理}
\newtheorem{lemma}{補題}
\newtheorem{example}{例}
\newtheorem{proposition}{命題}
\newtheorem{observation}{観察}
\newtheorem{problem}{問題}
\newtheorem{corollary}{系}
\newtheorem{remark}{注}
\newtheorem{fact}{事実}
\renewcommand{\proofname}{\bf 証明}
