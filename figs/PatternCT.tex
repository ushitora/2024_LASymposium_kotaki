\documentclass[dvipdfmx,border=10pt]{standalone}
\usepackage{tikz}

\begin{document}
% 変数の定義
\def\nodedist{5}          % ノード間の水平距離
\def\leveldist{10}         % レベル間の垂直距離
\def\nodesize{80}          % ノードの大きさ
\def\linewd{0.8}           % 線の太さ
\def\indexdist{3}          % インデックスラベルの距離

% 座標計算用のマクロ
\pgfmathsetmacro{\xzero}{0*\nodedist}   % node 0
\pgfmathsetmacro{\xone}{1*\nodedist}    % node 1
\pgfmathsetmacro{\xtwo}{2*\nodedist}    % node 2
\pgfmathsetmacro{\xthree}{3*\nodedist}  % node 3
\pgfmathsetmacro{\xfour}{4*\nodedist}   % node 4
\pgfmathsetmacro{\xfive}{5*\nodedist}   % node 5

% y座標の計算
\pgfmathsetmacro{\yone}{-1*\leveldist}    % レベル1
\pgfmathsetmacro{\ytwo}{-2*\leveldist}    % レベル2
\pgfmathsetmacro{\ythree}{-3*\leveldist}  % レベル3

    \begin{tikzpicture}[scale=1]
        % スタイルの定義
        \tikzset{
            tree node/.style={
                circle,
                draw=black,          % 線の色を明示的に指定
                line width=0.8pt,    % 線の太さを指定
                minimum size=\nodesize,
                inner sep=2pt,
                font=\Huge,
            },
            index label/.style={
                below=3pt,
                font=\large
            }
        }

        % ノードの定義
        \node[tree node] (n0) at (\xzero, \yone) {17};
        \node[tree node] (n1) at (\xone, \ythree) {25};
        \node[tree node] (n2) at (\xtwo, \ytwo) {21};
        \node[tree node] (n3) at (\xthree, 0) {11};     % ルート
        \node[tree node] (n4) at (\xfour, \ytwo) {27};
        \node[tree node] (n5) at (\xfive, \yone) {15};

        % エッジの描画
        \draw[line width=0.8pt] (n3) -- (n0);  % 11 -- 17
        \draw[line width=0.8pt] (n0) -- (n2);  % 17 -- 25
        \draw[line width=0.8pt] (n2) -- (n1);  % 25 -- 21
        \draw[line width=0.8pt] (n3) -- (n5);  % 11 -- 27
        \draw[line width=0.8pt] (n5) -- (n4);  % 27 -- 15

        % インデックスの表示(オプション)
        % \node[index label] at (n0) {0};
        % \node[index label] at (n1) {1};
        % \node[index label] at (n2) {2};
        % \node[index label] at (n3) {3};
        % \node[index label] at (n4) {4};
        % \node[index label] at (n5) {5};

    \end{tikzpicture}
\end{document}