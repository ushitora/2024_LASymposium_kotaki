\documentclass[dvipdfmx,border=10pt]{standalone}
\usepackage{tikz}

\begin{document}
% 変数の定義
\def\nodedist{5}          % ノード間の水平距離
\def\leveldist{3}         % レベル間の垂直距離
\def\nodesize{80}          % ノードの大きさ
\def\linewd{0.8}           % 線の太さ
\def\indexdist{3}          % インデックスラベルの距離

% 座標計算用のマクロ
\pgfmathsetmacro{\xone}{1*\nodedist}    % node 1
\pgfmathsetmacro{\xtwo}{2*\nodedist}    % node 2 (x座標)
\pgfmathsetmacro{\xthree}{3*\nodedist}  % node 3
\pgfmathsetmacro{\xfour}{4*\nodedist}   % node 4
\pgfmathsetmacro{\xfive}{5*\nodedist}   % node 5
\pgfmathsetmacro{\xsix}{6*\nodedist}    % node 6
\pgfmathsetmacro{\xseven}{7*\nodedist}  % node 7

% y座標の計算
\pgfmathsetmacro{\yone}{-1*\leveldist}    % レベル1
\pgfmathsetmacro{\ytwo}{-2*\leveldist}    % レベル2
\pgfmathsetmacro{\ythree}{-3*\leveldist}  % レベル3
\pgfmathsetmacro{\yfour}{-4*\leveldist}   % レベル4

    \begin{tikzpicture}[scale=1]
        % スタイルの定義
        \tikzset{
            tree node/.style={
                circle,
                draw=black,          % 線の色を明示的に指定
                line width=0.8pt,    % 線の太さを指定
                minimum size=\nodesize,
                inner sep=2pt,
                font=\Huge,
            },
            index label/.style={
                below=3pt,
                font=\large
            }
        }

        % ノードの定義
        \node[tree node] (n1) at (\xone, \ytwo) {17};
        \node[tree node] (n2) at (\xtwo, \yone) {10};
        \node[tree node] (n3) at (\xthree, \ythree) {19};
        \node[tree node] (n4) at (\xfour, \ytwo) {10};
        \node[tree node] (n5) at (\xfive, \yfour) {24};
        \node[tree node] (n6) at (\xsix, \ythree) {15};
        \node[tree node] (n7) at (\xseven, \yfour) {27};

        % エッジの描画
        \draw[line width=0.8pt] (n2) -- (n1);
        \draw[line width=0.8pt] (n2) -- (n4);
        \draw[line width=0.8pt] (n4) -- (n3);
        \draw[line width=0.8pt] (n4) -- (n6);
        \draw[line width=0.8pt] (n6) -- (n5);
        \draw[line width=0.8pt] (n6) -- (n7);


    \end{tikzpicture}
\end{document}