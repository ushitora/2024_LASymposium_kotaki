% beamer関連
\usepackage{bxdpx-beamer}
\usepackage{pxjahyper}
\usepackage{minijs}
\usefonttheme{professionalfonts}

% tikz
\usepackage{tikz}
\usetikzlibrary{automata, positioning, arrows.meta, bending, calc}

% その他
\usepackage[super]{cite}
\usepackage{efbox}
\usepackage{xcolor}
\usepackage[absolute,overlay]{textpos}
\usepackage[ruled, linesnumbered]{algorithm2e}
\usepackage{amsmath}
\usepackage{amssymb}
\usepackage{amsfonts}
\usepackage{xstring}
\usepackage{forloop}
\usepackage{graphicx}
\usepackage{multirow}
\usepackage{caption}
\usepackage{colortbl}


% 色の設定
\definecolor{green}{rgb}{0, 0.6, 0}

% ラベルの日本語化
\setbeamertemplate{theorems}[normal font] % イタリック体にしない
\uselanguage{japanese}
\languagepath{japanese}
\deftranslation[to=japanese]{Theorem}{定理}
\deftranslation[to=japanese]{Corollary}{系}
\deftranslation[to=japanese]{Lemma}{補題}
\deftranslation[to=japanese]{Example}{例}
\deftranslation[to=japanese]{Examples}{例}
\deftranslation[to=japanese]{Definition}{定義}
\deftranslation[to=japanese]{Definitions}{定義}
\deftranslation[to=japanese]{Problem}{問題}
\deftranslation[to=japanese]{Solution}{解}
\deftranslation[to=japanese]{Fact}{事実}
\deftranslation[to=japanese]{Proof}{証明}
\def\proofname{証明}
\deftranslation[to=japanese]{Figure}{図}

% \makeatletter
% \newsavebox{\@brx}
% \newcommand{\llangle}[1][]{\savebox{\@brx}{\(\m@th{#1\langle}\)}%
%   \mathopen{\copy\@brx\kern-0.5\wd\@brx\usebox{\@brx}}}
% \newcommand{\rrangle}[1][]{\savebox{\@brx}{\(\m@th{#1\rangle}\)}%
%   \mathclose{\copy\@brx\kern-0.5\wd\@brx\usebox{\@brx}}}
% \makeatother

% コマンド関連
\renewcommand{\kanjifamilydefault}{\gtdefault} % 既定をゴシック体にする
\renewcommand{\arraystretch}{0.8} % 表の縦幅を変更
% \renewcommand{\baselinestretch}{1.5} % 行間を変更
\renewcommand{\citeform}[1]{[#1]} % 上付きの参考文献引用
\renewcommand{\thefootnote}{\fnsymbol{footnote}}
\newcommand{\prev}[1]{{\langle #1 \rangle}_{\infty}}
\newcommand{\myprev}[1]{{\langle #1 \rangle_{\Pi}}}
\newcommand{\st}[0]{\text{s.t.}}
\newcommand{\Ninfty}{\mathbb{N}_{\infty}}
\newcommand{\arrangetext}[2]{
    \begin{tikzpicture}
        \foreach \text [count=\i from 0] in {#1} {
            \node [xshift=\i * #2mm, font=\strut\ttfamily, inner sep=0pt] at (\i, 0) {\text};
        }
    \end{tikzpicture}
}
\newcommand{\itabular}[2]{%
    \begin{tabular}{c}
        #1 \\
        #2
    \end{tabular}%
}
\newcommand{\lsf}[2]{%
    \mathit{lsf_{#1}\left(#2\right)}
}
\newcommand{\lsp}[2]{%
    \mathit{lsp_{#1}\left(#2\right)}
}
\newcommand{\Lsf}[3]{%
    \mathit{Lsf_{#1}\left(#2,~#3\right)}
}
\newcommand{\ELsf}[3]{%
    \mathit{Lsf_{#1}\left(\overleftarrow{#2},~\overrightarrow{#3}\right)}
}